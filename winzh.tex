%
%
% Win10、Win11自带中英文字体混排设置
% 中文使用系统自带华文字体(st-)
% 拉丁字符使用latinmodern系列
% (更新的sim-字体在某些机器上会出错,原因不明)
%
% 

% 定义后备字体家族
\definefallbackfamily[winzh][rm]            % 名称、类型
    [latinmodernroman]                              % 默认字体/字体家族
    [range={basiclatin,latinsupplement},force=yes]  % 强制用于西文
\definefallbackfamily[winzh][ss]            % 名称、类型
    [latinmodernsans]                              % 默认字体/字体家族
    [range={basiclatin,latinsupplement},force=yes]  % 强制用于西文
\definefallbackfamily[winzh][tt]            % 名称、类型
    [latinmodernmonolight]                              % 默认字体/字体家族 latinmodernmonolight latinmodernmonoproplight
    [range={basiclatin,latinsupplement},force=yes]  % 强制用于西文

% 定义字体家族
\definefontfamily [winzh][rm]   % 名称、类型: rm/serif, ss/sansserif, tt/teletype, hw/handwritten, cg/calligraphic, mm/mathematics
    [nsimsun]                           % 默认字体/字体家族
    [
        bf={stxihei},                    % 粗体
        it={stkaiti},                     % 斜体
        bi={stxinwei},                  % 粗斜体
    ]
\definefontfamily[winzh][ss][stfangsong][bf=stxihei,it=stkaiti,bi=stxinwei]
\definefontfamily[winzh][tt][stkaiti][bf=stxihei,it=stfangsong,bi=stxinwei]
% 拼音字体
% wukongpinyinsanslight

% 书写用田字格楷体、方格楷体
% 统编教材,一至四年级田字格1.45cm见方,五六年级方格1.25cm见方
\definefont[tiangekai] [name:stkaiti*default at 35pt]


% 官方文档:
% https://wiki.contextgarden.net/Command/definefontfamily
% https://wiki.contextgarden.net/Command/definefallbackfamily


% 来源于网上的一些示例(已经忘记出处):
% \definefontfamily[mainface]       [rm][TeX Gyre Pagella]
% \definefontfamily[changedstyles]  [rm][TeX Gyre Pagella][tf=style:italic,it=style:bold,bi=style:bolditalic,bf=style:regular]
% \definefontfamily[changedfiles]   [rm][TeX Gyre Pagella][it=file:texgyreherositalic,bi=file:texgyrecursorbolditalic]
% \definefontfamily[changedfeatures][rm][TeX Gyre Pagella][tf=features:smallcaps,bf=features:none]
% \definefallbackfamily [alegreya]  [rm][KaiseiOpti] [preset=range:japanese]
% \definefallbackfamily [studentA]  [ss][Noto Sans CJK SC][range={0x0000-0x058F},force=yes]%preset=range:chinese


%
%
% GB/T 18358—2009《中小学教科书幅面尺寸及版面通用要求》:
%
%
% a类   一至三年级      楷      21至16pt
% b类   二至四年级      楷、宋  14pt
% c类   五年级至高中    宋      12pt
% d类   高中理科        宋      10.5pt
