%
% 
% 一个用于生产的简中和英文混排字体设置。
% 宋、黑、西文用noto系列中、英字体,优点是:
% 工业级别;开源;大字符集;风格统一。
% 简中仿、楷没有工业级别的开源字体,
% 暂用方正、华文占位作为示例,请注意:
% !!!不是所有字体都有著作权
% !!!但没有著作权的字体文件也可能有知识产权
% !!!请自行保证合法使用
% 
%


% 宋体/衬线体家族
% 西文、数字强制使用后备字体家族,但尽可能优先使用主字族主字体
\definefallbackfamily[notoplus][rm]            % 名称、类型
    [Noto Serif]                              % 默认字体家族
    [%用默认字族替换it sl bi bs sc,而tf、bf用主字体
        tf={notoserifcjkscregular},%使用主字族字体
        bf={notoserifcjkscbold},%使用主字族字体
        it={notoseriflightitalic},%略细,默认则略粗
        range={basiclatin,latinsupplement},
        % check=yes,%查看主字体是否包含字符 yes no,会增加延迟;并得注意斜体等是不是正确
        force=yes,% 强制替换,即使主字体包含字符,这样设置更明确
        % scale=,
        % offset=,
    ]
% 非强制后备字体,用于缺字时
\definefallbackfamily[notoplus][rm]
    [notoserifcjksc]
    [
        it={stkaiti},
        preset=range:chinese,
    ]
\definefontfamily [notoplus][rm]   % 名称、类型: rm/serif, ss/sansserif, tt/teletype, hw/handwritten, cg/calligraphic, mm/mathematics
    [notoserifcjksc]                           % 默认字体/字体家族
    [
        bf={notosanscjkscregular},        % 粗体用黑体表示
        it={fzkaiz03s},                     % 斜体用楷体表示
        bi={notosanscjkscbold},           % 粗斜体用粗黑表示
    ]


% 等线/无衬线家族
\definefallbackfamily[notoplus][ss]
    [Noto Sans]
    [
        tf={notosanscjksclight},
        bf={notosanscjkscmedium},
        range={basiclatin,latinsupplement},
        force=yes,
    ] 
\definefallbackfamily[notoplus][ss]
    [notosanscjksc]
    [
        it={stkaiti},
        preset=range:chinese,
    ]
\definefontfamily[notoplus][ss]
    [notosanscjksc]
    [
        tf={notosanscjksclight},
        bf={notosanscjkscmedium},
        it={fzkaiz03s},
        bi={notosanscjkscbold}
    ]

% 仿宋/打字家族 fzfangsongz02s  stfangsong
\definefallbackfamily[notoplus][tt]
    [Noto Serif]
    [%比照宋体家族而字重降低两级 https://source.typekit.com/source-han-serif/cn/
        tf={fzfangsongz02s},
        bf={notoserifcjkscmedium},
        range={basiclatin,latinsupplement},
        force=yes,
    ]
\definefallbackfamily[notoplus][tt]
    [stfangsong]
    [
        it={stkaiti},
        preset=range:chinese,
    ]
\definefontfamily[notoplus][tt]
    [fzfangsongz02s]
    [
        bf={notosanscjkscregular},
        it={fzkaiz03s},
        bi={notosanscjkscbold}
    ]


% 楷体家族 stkaiti  fzkaiz03s
\definefallbackfamily[notoplus][hw]
    [Noto Sans]
    [
        tf={notosanscjksclight},
        bf={notosanscjkscmedium},
        range={basiclatin,latinsupplement},
        force=yes,
        scale=0.85,
    ] 
\definefallbackfamily[notoplus][hw]
    [stkaiti]
    [
        it={stfangsong},
        preset=range:chinese,
    ]
\definefontfamily[notoplus][hw]
    [fzkaiz03s]
    [
        bf={notosanscjkscregular},
        it={fzfangsongz02s},
        bi={notosanscjkscbold}
    ]


% ConTeXt官方文档:
% https://wiki.contextgarden.net/Command/definefontfamily
% https://wiki.contextgarden.net/Command/definefallbackfamily


% noto字体信息
% https://fonts.google.com/


% 来源于网上的一些示例(已经忘记出处):
% \definefontfamily[mainface]       [rm][TeX Gyre Pagella]
% \definefontfamily[changedstyles]  [rm][TeX Gyre Pagella][tf=style:italic,it=style:bold,bi=style:bolditalic,bf=style:regular]
% \definefontfamily[changedfiles]   [rm][TeX Gyre Pagella][it=file:texgyreherositalic,bi=file:texgyrecursorbolditalic]
% \definefontfamily[changedfeatures][rm][TeX Gyre Pagella][tf=features:smallcaps,bf=features:none]
% \definefallbackfamily [alegreya]  [rm][KaiseiOpti] [preset=range:japanese]
% \definefallbackfamily [studentA]  [ss][Noto Sans CJK SC][range={0x0000-0x058F},force=yes]%preset=range:chinese


%
%
% GB/T 18358—2009《中小学教科书幅面尺寸及版面通用要求》:
%
%
% a类   一至三年级      楷      21至16pt
% b类   二至四年级      楷、宋  14pt
% c类   五年级至高中    宋      12pt
% d类   高中理科        宋      10.5pt


% 拼音字体
% \definefont[pinyin] [name:wukongpinyinsanslight*default at 10.5pt]

% 书写用田字格楷体、方格楷体
% 统编教材,一至四年级田字格1.45cm见方,五六年级方格1.25cm见方
% \definefont[tiangekai] [name:stkaiti*default at 35pt]
